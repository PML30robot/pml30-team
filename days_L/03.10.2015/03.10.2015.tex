\subsubsection{03.10.2015}
	\textit{\textbf{Time frame:}} 16:00-21:30 \newline
	\textit{\textbf{Preview:}} The purpose for current meeting was to divide all construction works into 4 groups (one group for one teammate) to provide elaborating modules in parallel. After that, we wrote technical specification for each group of modules to help collaborators follow the requirements. \newline \newline
	\textit{\textbf{Technical specifications for modules:}}
  \begin{enumerate*}
  	\item Chassis
  	\begin{enumerate*}
  		\item Carriage consists of two lengthwise beams 41.5cm connected at the back side. All other modules will be mounted to this base. 
  		
  		\item Wheel base consists of 3 pairs of standard wheels. All wheels at one side are linked to each other and move dependently.
  		
  		\item Wheel base is powered by 6 dc motors (3 at one side). 
  		
  		\item Motors should not interfere with bucket which will be placed in the forward half of the robot. 
  		
  		\item There should be no construction elements except wheels that can touch the surface while climbing the ramp. 
  		
  	\end{enumerate*}
  	
  	\item Elevator
  	\begin{enumerate*}
  		\item Elevator consists of 2 static beams fixed on the base and one pivoting with bucket on it.
  		
  		\item Turnable beam rotates around the axis on top of the static beams. It's rotation powered by 1 dc motor with gear ratio 1:3.
  		
  		\item Length of the elevator should be enough for scoring debris into all boxes without climbing to the 3-rd zone.
  	\end{enumerate*}
  	
  	\item Bucket
  	\begin{enumerate*}
  		\item Bucket should be fixed to the turning beam of elevator stationary.
  		
  		\item Free space inside the bucket should be 10-14cm at width, 15-17cm in length and 7cm in height. It should be capacious enough for containing 5 cubes of 3 balls.
  		
  		\item The back side of the bucket can be narrower to prevent collecting more than 5 cubes at once (cubes will settle as $2+2+1$). 
  		
  		\item Bucket's movement should not interfere with gripper for debris.
  		
  		
  		\item Entrance hole of the bucket should have the same height and width, as the internal space.
  		
  		\item Bucket should have a turning flap above the entrance which can prevent balls from scoring on demand. Additionally, the flap will stop debris from falling out of the bucket when it will be overturned.
  	\end{enumerate*}
  	
  	\item Gripper
  	\begin{enumerate*}
  		\item Gripper consists of 2 rotating blades, mounted to axis at an angle $180^{\circ}$ to each other.
  		
  		\item Gripper is powered by 1 or 2 continious rotating servos.
  		
  		\item Gripper is placed ahead the bucket. Width of blades should match with the entrance of the bucket.
  		
  		\item Space between axis and field enough for unhindered passage of balls.
  		
  		\item Gripper should not make any obstacles for bucket's moving.
  		
  		\item At both sides of the blade's working area placed slopes, which are tapering to the bucket.
  	\end{enumerate*}
  	
  	\item Alpinists
  	\begin{enumerate*}
  		\item Mechanism for scoring autonomus alpinists will be placed at the forward side of robot. It's definite position will be determined after discussion of autonomus strategy.
  		
  		\item Mechanism consists of L-shaped beam powered by standard servo.
  		
  		\item At the end of beam placed a bucket for 2 alpinists.
  		
  		\item Module should not interfere with gameplay after the autonomus period ends.
  	\end{enumerate*}
  	
  \end{enumerate*}
  
   \newline
  \textit{\textbf{Responsibilities for each module:}}
  \begin{enumerate*}
  	\item Carriage and wheel base - Victoria Loseva
  	
  	\item Bucket and elevator - Evgeniy Maksimychev
  	
  	\item Gripper with slopes - Ivan Afanasiev
  	
  	\item Mechanism for scoring alpinists - Nikita Safronov
  \end{enumerate*}
  
   \newline
  \textit{\textbf{Additional comments:}} Now our team is ready to proceed to the next objective: designing modules.

\fillpage
