\addtocounter{number_of_meeting}{1}
\subsubsection{Concept discussing (24.09 - 27.09)}
\textit{\textbf{Time frame:}} 24.09 - 27.09 \newline
\textit{\textbf{Preview:}} The main purpose for current meeting was to figure out how the modules of the robot should look and how they will be developed. \newline \newline
\textit{\textbf{Modules:}}

\begin{table}[H]
	\vspace{-2mm}
	\begin{center}
		\begin{tabular}{|p{0.2\linewidth}|p{0.7\linewidth}|p{0.1\linewidth}|}
			\hline
			Modules & Solutions & Label \\
			\hline
			Wheel base & Six standard wheels & wheel base \\
			\hline
			Lift & Retracktable rails & lift \\
			\hline
			Gripper & Rotating brush and bucket & gripper\\
			\hline
			Scoring autonomous climber and pushing button & F - shaped beam & climbers + button\\
			\hline
			Scoring climbers in tele op & Retracktable slat & climbers\\
			\hline
			Pulling & Motor that reel the rope & pullup\\
			\hline
			Push the clear signal & Servo with beam & clear signal\\
			\hline
		\end{tabular}
	\end{center}
\end{table}

\newline
\textit{\textbf{Detailed explaination:}}
\begin{enumerate}
	\item Wheel base will consists of six standard wheels which rotates with help of six DC motors. It allows to climb to low and middle zone fast enough.
	
	\item We decided to use the lift in our robot. It help us to score elements to high goal from low or middle zone. So we don't need climb to the high zone. We choosed the construction with inclined retracktable construction profiles because it is the most simple and relliable.
	
	\item The robot will collect elements with help of rotating brush which pull them to the special bucket which connected with the lift. This method is the most simple and fast. After collecting elements the bucket rises by the lift. Then it overturns to the side and elements fall to the box.
	
	\item We decided to make one mechanism for scoring autonomous climbers and pushing the button. It is the F-shaped beam. In the top beam is the bucket for climbers, in the bottom - axle which push button. When we turn this mechanism the axle push the button and in the same time climbers fall to the goal.
	
	\item For scoring climbers in tele op we decided to use horizontal retracktable slat that move to the both sides by the whhel that rotate with help of servo of continuous rotation. When the slat extract it push to the hook that fix zip line.
	
	\item The pulling mechanism is the 2 DC motors that reel the rope which connected with the hook that fixed on the lift. Also this motors rise the lift. When lift is rising the rope is extracting. When the robot pull up rope is reeling and lift is lowering. 
	
	\item For pushing clear signal we decided use the servo with beam that fixed on the lift.

	
\end{enumerate}

\newline
%\textit{\textbf{Additional comments:}} 


\fillpage
