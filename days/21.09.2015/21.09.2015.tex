\subsection{Brainstorming}

\addtocounter{number_of_meeting}{1}
\subsubsection{\arabic{number_of_meeting} Meeting}
	\textit{\textbf{Time frame:}} 22.09.2015 17:00-21:00 \newline
	\textit{\textbf{Preview:}} Since this year FTC rules were published, every member of our team had carefully read them. Today we gathered together to discuss all the aspects of this year gameplay and think of how to get on with the most significant features of the game. \newline
	
	 \newline
	\textit{\textbf{General aspects:}}
	\begin{table}[H]
		\vspace{-2mm}
		\begin{center}
			\begin{tabular}{|p{0.4\linewidth}|p{0.5\linewidth}|p{0.1\linewidth}|}
				\hline
				Features & Solutions & Label \\
				\hline
				Moving to the ramp is essential if we want to achieve high score. & We need to realise the wheel base that will be good at moving on the ramp. & chassis \\
				\hline
				The it will take a lot of time to climb to the 3-rd zone of the ramp. & We can deliver debris to the highest goal with elevator instead of climbing in driver-control period. & elevator \\
				\hline
				Space between each two bars in 3-rd zone is wider than the standard TETRIX wheel diameter. & We can use tracks or 3-4 wheels from each side of the robot in order to not to get stuck. & chassis \\
				\hline
				Goals for debris have a very little capacity. & It is more preferable to collect cubes than balls. That's why we need mechanism to prevent balls from collecting. & capture \\
				\hline
				Pulling up costs 80 points. It's not difficult to realise then. & We need to spend at least 1 DC motor for pulling up. We can grasp the pull-up bar with hook and lift to it by reeling the cable. & pull up \\
				\hline
				Moving over the inclined plane and pulling up require high moment on motors. However, the number of motors is limited. & Our robot should be light enough to decrease the moment required for moiving and, as a result, increase speed of moving. & weight \\
				\hline
				It's quite unconvenient to exchange ramps with your ally during the game. & We will negotiate with our ally about spheres of influence before each game. We need to make two autonomus programs for climbing onto both ramps. & strategy \\
				\hline
				Robot can grip 5 debris at once, when the maximal capacity of one bucket is 24 cubes. So, to fill one bucket robot has to repeat collecting and taking cubes to the goal 5 times per 1,5 minutes & We can make gripper for debris at the front side of the robot and extract scoring elements from the back side. It will allow us to go to the ramp backwards, so we won't need to turn around on the ramp before going down to collect debris and save time. & concept \\
				\hline
				All the zones of red alliance are the mirror reflection of blue alliance's zones. & Our robot should be symmetrical and capable of playing on both sides of field. & concept \\
				\hline
				This year autonomus period has no difficult tasks. The only hardness is that both robots in alliance have to fulfil the same tasks at the same place. Furthermore, robots can start autonomus period form different positions. So, it's difficult to predict how the another robot in our allianse will move. & We need to make a number of programs for autonomus period from different positions for easier adjustment to the ally's strategy. & strategy \\
				\hline
				It's not restricted to collect debris in autonomus period. & We need to realise automatically collection of 5 cubes in autonomus period. At the conclusion of autonomus period the robot will remain on the ramp with 5 cubes and we will put them to the goal immediately & strategy \\
				\hline
			\end{tabular}
		\end{center}
	\end{table}
	
	 \newline
	\textit{\textbf{The main conception of engineering process:}} FTC rules have a various number of heterogeneous objectives. Some of them are simple, while other are quite challenging. The quality of performance in same tasks depends on laboriousness of realisation of mechanisms. \newline
	In these conditions, we made a decision to develop two versions of robot:
	\begin{enumerate*}
		\item a simpe, but reliable one, to startup and perform in regional competition
		
		and
		
		\item a high-quality one, which will take a lot of time to design and assemble to perform in further competitions.
		
	\end{enumerate*}
	
	 \newline
	\textit{\textbf{Detailed explaination:}}
	\begin{enumerate*}
		\item Detailed explaination of robot...
		\begin{figure}[H]
			\begin{minipage}[h]{1\linewidth}
				\center{\includegraphics[scale=0.2]{00.00.2015/images/01}}
				\caption{robot}
			\end{minipage}
		\end{figure}
		
		\item Detailed explaination of program...
		\begin{figure}[H]
			\begin{minipage}[h]{1\linewidth}
				\center{\includegraphics[scale=0.2]{00.00.2015/images/02}}
				\caption{robot}
			\end{minipage}
		\end{figure}
		
	\end{enumerate*}
	
	 \newline
	\textit{\textbf{Additional comments:}} For the next meeting we need to think of two issues:
	\begin{enumerate*}
		\item which tasks our simple robot should be able to execute without loss of efficiency
		
		and
		
		\item to set the priorities of performing tasks during the game.
		
	\end{enumerate*}
  


\fillpage
