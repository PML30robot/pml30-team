\subsubsection{28.12.2015}
\textit{\textbf{Time frame:}} 16:00-22:00

Today the operators were practicing in driving the robot.

During the trainings it was found out that the ramp for debris scrabs the field. So, it was moved up.

It was also investigated that the robot will not overturn while climbing to the mountain if the elevator is extracted on about 30 cm or more. However, after the wheels stop moving, the robot slips down from the mountain. So, it was essential to make a mechanism for grasping the lower hurdle of the second zone to stay at the mountain.

It was revealed that the brushes are turning in opposite direction. So, it was decided to install once more gear to inverse the direction of rotation of the first brush.

The mechanism for shifting the bucket and the servos for overturning the bucket and opening the cover were tested. Both of these systems worked fine.

It was investigated that the power of 2 standard TETRIX DC motors is nit enough to extract the elevator to the full height. It was the reason why one of the motors broke down.

It was decided to install 3 motors instead of 2 and replace standard TETRIX motors with AndyMark motors because AndyMark motors are more reliable concerning to stalling.

One of the themes of this meeting was discussion of working plans for winter holidays. As a result of it was made the following table of tasks.
\begin{table}[H]
	\caption{Results of discussion of holidays working plan}
	\label{tabular:meetingSPB28.12}
	\begin{center}
		\begin{tabular}{c|c}
		  Task & Responsible \\
		  \hline
		  Study out why does the lift's motors break & Nikita \\
		  Test bucket & Nikita \\
		  Make crossbar engagements	& Nikita \\
		  Finish the grab & Andrey \\
		  Think of mechanism for autonomous climbers & Andrey \\
		  Create autonomous programm & Andrey \\
	    \end{tabular}
	\end{center}
\end{table}

