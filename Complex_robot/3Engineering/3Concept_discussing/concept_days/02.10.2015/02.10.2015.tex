\subsubsection{02.10.2015}
	\textit{\textbf{Time frame:}} 17:00-19:30 \newline
	\textit{\textbf{Preview:}} The purpose of this meeting was to divide all construction works into 4 groups (one group for one teammate) to elaborate modules in parallel. After that, we wrote the technical specifications for each group of modules to help collaborators follow the requirements. \newline \newline
	\textit{\textbf{Technical specifications for modules:}}
  \begin{enumerate*}
  	\item Chassis
  	\begin{enumerate*}
  		\item Carriage consists of two lengthwise beams 41.5cm connected at the back. All other modules will be mounted to this base. 
  		
  		\item Wheel base consists of 3 pairs of standard wheels. All wheels at one side are linked to each other and move together.
  		
  		\item Wheel base is powered by 4 dc motors (2 at one side). 
  		
  		\item Motors should not interfere with the bucket, which will be placed in the front half of the robot. 
  		
  		\item While the robot is climbing the ramp, no construction elements but the wheels should be touching the surface of the ramp. 
  		
  	\end{enumerate*}
  	
  	\item The mechanism that turns the elevator
  	\begin{enumerate*}
  		\item A continuous rotation servo will turn the worm gear.
  		
  		\item It should be mounted on the side beam of the base.
  	\end{enumerate*}
  	
  	\item Elevator
  	\begin{enumerate*}
  		\item Elevator consists of retracktable construction profiles which connected with help of special elements. The shape and size of these elements should be fit with grooves in profiles.
  		
  		\item It should be mounted on the turning mechanism.
  		
  		\item Length of the elevator should be enough for scoring debris into high and middle boxes from low zone and starting pullup from the middle zone. 
  		
  		\item A thread and block system will provide lifting of elevator.
  		
  		\item The servo that turn clear signal should be fixed on the top of the elevator.
  		
  		\item The hook for pulling the robot up will also be mounted on the top of the elevator.
  		
  	\end{enumerate*}
  	
  	\item Bucket
  	\begin{enumerate*}
  		
  		\item The bucket will be fixed to a beam turned by a servo on the top of the lift.
  		
  		\item Free space inside the bucket should be 10-14cm at width, 15-17cm in length and 7cm in height. It should be spacious enough to contain 5 cubes of 3 balls.
  		
  		\item To prevent gathering more than five cubes at once, the bucket will narrow down to the back (cubes will settle as $2+2+1$). 
  		
  		\item The bucket's movement should not interfere with debris gripper.
  		
  		
  		\item The entrance hole of the bucket should have the same height and width as the internal space.
  		
  		\item Bucket should have a turning flap above the entrance which can prevent balls from scoring not on demand. Additionally, the flap will stop debris from falling out of the bucket when it is be flipped over.
  	\end{enumerate*}
  	
  	\item Gripper
  	\begin{enumerate*}
  		\item Gripper consists of 2 rotating blades which form a $180^{\circ}$ angle.
  		
  		\item Gripper is powered by 1 or 2 continiously rotating servos.
  		
  		\item Gripper is placed in front the bucket. Blade width should match the bucket entrance.
  		
  		\item Space between axis and field is enough for unhindered passage of balls.
  		
  		\item Gripper should not pose any obstacle for bucket motion.
  		
  		\item At both sides of the blade's working area placed slopes, which are tapering to the bucket.
  	\end{enumerate*}
  	
  	\item Scoring autonomous climbers + pushing button
  	\begin{enumerate*}
  		\item The mechanism for scoring autonomus alpinists will be placed at the front right side of robot. It's definite position will be determined after discussion of autonomus strategy.
  		
  		\item Mechanism consists of F-shaped beam powered by standard servo.
  		
  		\item At the end of top beam is a bucket for 2 alpinists. The bottom beam pushes the button.
  		
  		\item Module should not interfere with gameplay after the autonomus period ends.
  	\end{enumerate*}
  	
  	\item Mechanism for extracting lift and pulling
  	
  	\begin{enumerate*}
  		\item Two reels that are rotated by 4 DC motors.
  		
  		\item The rope for pulling and line for extracting lift are in different reels. When the line wound the rope unwound and in other way.
  		
  		\item It should be mounted on the back beam of the base.
  		
  	\end{enumerate*}
  	
  \end{enumerate*}
  
   \newline
  \textit{\textbf{Responsibilities for each module:}}
  \begin{enumerate*}
  	\item Carriage and wheel base - Gordei Kravtsov
  	
  	\item Bucket and mechanism for shifting it - Aleksandr Iliasov
  	
  	\item Elevator and winch - Nikita Safronov
  	
  	\item Gripper with slopes and the mechanism for scoring alpinists - Andrew Nemov
  	
  	\item Mechanism for scoring alpinists - Anton Ponikarovskiy
  \end{enumerate*}
  
   \newline
  \textit{\textbf{Additional comments:}} Now our team is ready to proceed working on next objective: designing modules.

\fillpage
