\subsubsection{Elevator}
  
  Engineering tasks included in this module:
  \begin{enumerate*}
    \item Lifting mechanism 
    \begin{enumerate*}
       \item According to the technical plan our team created beforehand, the lifting mechanism was to consist of several construction beams connected to each other with special parts. To create these parts, we first thought through the concept and 3d-modeled them in Creo Parameteric 3.0. These parts are something akin to a brace and will stabilize one beam in relation to another. There are two types of braces: for the central gaps, and side gaps of the beams. Let us consider the simplest way to connect the two beams with these braces. Beam A will be fixed in place to the base of the robot, and beam B will be fixed relative to beam A. Then we can can connect the first three braces to the top of beam A and the second three braces to the bottom of beam B, allowing for maximum freedom of movement for one of the beams against the other. For greater stability we use two groups of three on each end of the beams. 
       
       \item We must find such a height and length of the lifting mechanism that there would be an angle of tilt that would allow the robot to throw debris into the highest and middle bucket goals from the low zone, and grab the pull-up bar from the middle zone. 
       
       \item Knowing that the individual beams are 350mm long, we calculated that in order to reach this height, we need four beams.  
       
       \item Lifting four beams requires a block system - e.g. we need to add blocks with twine that, when reeled in, would lift the system. 
       
       \item In order to fasten the braces we need to add caps on the end faces of the beams. I came to the conclusion that I needed to change the caps: drill a hole through their legs, so that the twine could be put through them, and grind off the heads somewhat in order to make a trough, through which to pass the line. This allowed me to avoid adding additional blocks on the beams. 
    \end{enumerate*}
    \item Turning mechanism
    \begin{enumerate*}
       \item The turning mechanism consists of a servomotor attached to the base of the robot, a worm on the axis, and a gear on the first beam of the lifting mechanism. The servomotor turns the worm, which, in turn, rotates the gear, and the lifting mechanism tilts. 
    \end{enumerate*}
    
    \item Reeling mechanism
    \begin{enumerate*}
    	\item The reeling mechanism is a system of two coils powered by 2-4 motors. It is both used for extracting elevator and pulling up. The principle of work is folowing: one coil pulls the cable and extracts the elevator when the another coil releases strong cable used for pulling up. When the elevator is fully put forward, so does the pullup cable. Next, when the coil rotates backwards, it pulls the pullup cable and releases elevator's cable causing it to fold back.
    \end{enumerate*}
  
  \end{enumerate*}	
  
  %Because of the first competition was very close and 
  
  