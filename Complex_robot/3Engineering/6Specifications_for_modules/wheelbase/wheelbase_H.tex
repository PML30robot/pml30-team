\subsubsection{Carriage and wheel base}
  
  \begin{enumerate*}
    \item The current module has two functions. At first, it provides movement of the robot, and secondly it carries all the other modules. Due to this, it was divided into two submodules: a wheel base, and a frame.
    
    The frame should have 2 lengthwise beams of 41.5 cm, which will be used for mounting wheel base. These beams should be reliably connected to each other by the cross beams. The cross beams can be installed only at the back side of the robot, because at the front side there should be a gripper for debris and a bucket. Above the gripper it can be installed an additional cross beam, but this connection won't be strong enough as it will be placed far from lengthwise beams. That's why it was decided to apply 2 main cross beams as it depicted in the figure 1. 
    
    In the wheel base it was decided to use the construction with 6 standard 10 cm wheels. Firstly, the wheels are more reliable, than tracks, because tracks can run over during the match, and in this case the robot would be disabled. Secondly, the constriction with 6 wheels provides better rotation than 4 and 8 standard wheels (explanation at page ...). Omni wheels can't be used in this construction because they're working unstable at the ramp. 
    The decision of using 10 cm wheels instead of 7 cm is caused by two reasons: firstly, they're more convenient to install so that the robot will be capable of climbing the ramp (figure 2). Secondly, the middle sized gears and chain gears have sizes a bit less than 7 cm, so if they will be used with 7 cm wheels they can scratch the field if the robot will bend the floor covering too much.
    
    It was decided to use 4 NeveRest motors by AndyMark for movement, because these motors are more reliable than TETRIX DC motors. Due to calculations, it was revealed, that to climb the ramp with 10 cm wheels the gear ratio 2:1 on motors is not enough, so it was decided to apply the gear ratio 1:1.
    
    It was decided to use middle sized gears for connecting wheels instead of chain, as the chain requires more space and also inconvenient for connecting axes that stand along the one line.
    
    On the October 12$^th$ it was assembled the first prototype of the chassis. It had 7 cm wheels and a gear ratio for speed (2:1). It's test drive showed that 7 cm wheels can't be applied. 
    
    On the October 19$^th$ it was assembled a second version of the chassis. Each side consists of 3 wheels and 5 middle sized gears. Motors are placed at the back, so the space between 3 forward pairs of gears is free. 3 forward axes are fixed on the 16 cm beam, so it is easy to remove them if it is needed. 
    
    On the October 23$^th$, when the assembling of the winch started, it was found out, that the cross beams waste too much space inside the robot and there is not enough space for the winch. Since then, the former cross beams were replaced with more compact construction.
    
    At the first competition (...) it was investigated, that back DC motors are fixed to the frame not strong enough, so they unscrew very often. This problem was solved on .... The motor's mounts were fixed with more screws and with washers. 
    
    On ... there was installed the mount for the winch above the gripper. It also strethened the frame of the robot.
  
  \end{enumerate*}
  
  \fillpage