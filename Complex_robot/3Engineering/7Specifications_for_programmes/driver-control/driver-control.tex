\subsubsection{Driver control program}
	
	\begin{enumerate}
	    \item As soon as the first prototype of the wheel base was assembled on November $12^\textbf{th}$, it was elaborated a program for test-drive. It included straight movement and turning around in 4 grades of speed. With this program, there were tested the abilities of the present wheel base. 
	    
	    \item Results of the test drive were analysed so as develop a convenient control system. At first, turning around on high speed is inaccurate. So, the speed of turn was reduced proportionally to speed of straight movement. There also were added extra active buttons for accurate movement. Main drive control was moved from TopHat to a left stick. The operating area of the stick was divided into 8 zones. Zones 3 and 5 %(fig. 1) 
	    are not used because of inconvenience of back semi-turns. 
	    
	    \item Due to testing it was discovered, that optimal course speed to turn speed proportion varies non-linearly from one speed mode to another. So, it's more preferable to set speed mode by exact values of both speed parameters instead of common coefficient. In addition, it was decided to reduce the number of sectors on main stick's from 8 to 6 because 2 sectors were not in use %(fig. 2) 
	    .
	    
	    \item After the program for movement was finished, the controlling of the hooks and the debris collecting mechanism was placed to free buttons on driver-operator's joystick. Controlling of other modules was given to the second operator.
	    
        \item Before we replaced the NXT brick with the mobile phone the program was rewritten in Android Studio. So, when the new controlling system was installed (2 weeks before the competition in Sochi), we had the program adopted to it.
  \end{enumerate}
  
  
\fillpage
