\subsubsection{Driver control program}
	
	\begin{enumerate*}
	    \item As soon as the first prototype of the wheel base was assembled on November $12^\textbf{th}$, it was elaborated a program for test-drive. It included straight movement and turning around in 4 grades of speed. With this program, there were tested the abilities of the present wheel base. \newline
	    Here is the source code of first version:\newline
	    {\small
	    \begin{verbatim}
	    #pragma config(Hubs,  S1, HTMotor,  HTMotor,  none,     none)
	    #pragma config(Sensor, S1,     ,               sensorI2CMuxController)
	    #pragma config(Sensor, S2,     TIR,            sensorI2CCustom)
	    #pragma config(Motor,  mtr_S1_C1_1,     LF,            tmotorTetrix, openLoop, reversed)
	    #pragma config(Motor,  mtr_S1_C1_2,     LB,            tmotorTetrix, openLoop, reversed)
	    #pragma config(Motor,  mtr_S1_C2_1,     RF,            tmotorTetrix, openLoop)
	    #pragma config(Motor,  mtr_S1_C2_2,     RB,            tmotorTetrix, openLoop)
	    //*!!Code automatically generated by 'ROBOTC' configuration wizard               !!*//
	    
	    #include "JoystickDriver.c"
	    int k = 1;
	    int speed = 12;
	    
	    void movement()
	    {
	    if (joy1Btn(1))	// 1 speed
	    {
	    k = 1;
	    }
	    if (joy1Btn(2))	// 2 speed
	    {
	    k = 2;
	    }
	    if (joy1Btn(3))	// 3 speed
	    {
	    k = 4;
	    }
	    if (joy1Btn(4))	// 4 speed
	    {
	    k = 8;
	    }
	    if (joystick.joy1_TopHat == 6) // tank turn to left
	    {
	    motor[RB] = motor[RF] = speed * k;
	    motor[LB] = motor[LF] = -speed * k;
	    }
	    if (joystick.joy1_TopHat == 2)	// tank turn to right
	    {
	    motor[RB] = motor[RF] = -speed * k;
	    motor[LB] = motor[LF] = speed * k;
	    }
	    if (joystick.joy1_TopHat == 0)	// forward
	    {
	    motor[RB] = motor[RF] = speed * k;
	    motor[LB] = motor[LF] = speed * k;
	    }
	    if (joystick.joy1_TopHat == 4)	// backward
	    {
	    motor[RB] = motor[RF] = -speed * k;
	    motor[LB] = motor[LF] = -speed * k;
	    }
	    if (joystick.joy1_TopHat == -1)	// stop
	    {
	    motor[RB] = motor[RF] = 0;
	    motor[LB] = motor[LF] = 0;
	    }
	    }
	    
	    task main()
	    {
	    while(true)
	    {
	    getJoystickSettings(joystick);
	    movement();
	    wait1Msec(50);
	    }
	    }
	    
	    \end{verbatim} 
	    }
	    % % % % % % % % % % %
	    \item Results of the test drive were analysed so as develop a convenient control system. At first, turning around on high speed is inaccurate. So, the speed of turn was reduced proportionally to speed of straight movement. There also were added extra active buttons for accurate movement. Main drive control was moved from TopHat to a left stick. The operating area of the stick was divided into 8 zones. Zones 3 and 5 %(fig. 1) 
	    are not used because of inconvenience of back semi-turns. \newline
	    Here is the source code of second version: \newline
	    {\small
	    	\begin{verbatim}
	    	
	    	#include "JoystickDriver.c"
	    	int k = 2;
	    	
	    	int joy_result()
	    	{
	    	int horison_in = joystick.joy1_x1;
	    	int vertical_in = joystick.joy1_y1;
	    	
	    	int option = -1;
	    	int sector = 0;
	    	
	    	//determining sector
	    	if (vertical_in > 90) sector += 1;
	    	if (vertical_in < -90) sector += 4;
	    	if (horison_in > 100) sector += 2;
	    	if (horison_in < -100) sector += 8;
	    	
	    	//set option  
	    	if (sector == 1) option = 0;
	    	if (sector == 3) option = 1;
	    	if (sector == 2) option = 2;
	    	if (sector == 6) option = 3;
	    	if (sector == 4) option = 4;
	    	if (sector == 12) option = 5;
	    	if (sector == 8) option = 6;
	    	if (sector == 9) option = 7;
	    	
	    	return option;
	    	}
	    	
	    	int setspeed()
	    	{
	    	if (joy1Btn(1))	// 1 speed
	    	{
	    	k = 1;
	    	}
	    	if (joy1Btn(2))	// 2 speed
	    	{
	    	k = 2;
	    	}
	    	if (joy1Btn(3))	// 3 speed
	    	{
	    	k = 4;
	    	}
	    	if (joy1Btn(4))	// 4 speed
	    	{
	    	k = 8;
	    	}
	    	
	    	return k;
	    	}
	    	
	    	void joy_accomplish()
	    	{
	    	int speed = 12;
	    	int turnspeed = 6;
	    	int option = joy_result();
	    	int k = setspeed();
	    	
	    	if (option == 0)	// forward
	    	{
	    	motor[RB] = motor[RF] = speed * k;
	    	motor[LB] = motor[LF] = speed * k;
	    	}
	    	if (option == 4)	// backward
	    	{
	    	motor[RB] = motor[RF] = -speed * k;
	    	motor[LB] = motor[LF] = -speed * k;
	    	}
	    	if (option == 6) // tank turn to left
	    	{
	    	motor[RB] = motor[RF] = turnspeed * k;
	    	motor[LB] = motor[LF] = -turnspeed * k;
	    	}
	    	if (option == 2)	// tank turn to right
	    	{
	    	motor[RB] = motor[RF] = -turnspeed * k;
	    	motor[LB] = motor[LF] = turnspeed * k;
	    	}
	    	if (option == 7) // semi turn to left
	    	{
	    	motor[RB] = motor[RF] = speed * k;
	    	motor[LB] = motor[LF] = 0;
	    	}
	    	if (option == 1)	// semi turn to right
	    	{
	    	motor[RB] = motor[RF] = 0;
	    	motor[LB] = motor[LF] = speed * k;
	    	}
	    	if (option == -1)	// stop
	    	{
	    	motor[RB] = motor[RF] = 0;
	    	motor[LB] = motor[LF] = 0;
	    	}
	    	}
	    	
	    	void accurate_turn()
	    	{
	    	if (joy1Btn(5) || joy1Btn(6) || joy1Btn(7) || joy1Btn(8))
	    	{
	    	int speed = 24;
	    	int turnspeed = 12;
	    	int timeDelay = 100;
	    	
	    	if (joy1Btn(8))	// accurate forward
	    	{
	    	motor[RB] = motor[RF] = speed;
	    	motor[LB] = motor[LF] = speed;
	    	wait1Msec(timeDelay);
	    	}
	    	if (joy1Btn(7))	// accurate backward
	    	{
	    	motor[RB] = motor[RF] = -speed;
	    	motor[LB] = motor[LF] = -speed;
	    	wait1Msec(timeDelay);
	    	}
	    	if (joy1Btn(6))	// accurate right
	    	{
	    	motor[RB] = motor[RF] = -turnspeed;
	    	motor[LB] = motor[LF] = turnspeed;
	    	wait1Msec(timeDelay);
	    	}
	    	if (joy1Btn(5))	// accurate left
	    	{
	    	motor[RB] = motor[RF] = turnspeed;
	    	motor[LB] = motor[LF] = -turnspeed;
	    	wait1Msec(timeDelay);
	    	}
	    	
	    	motor[RB] = motor[RF] = 0;
	    	motor[LB] = motor[LF] = 0;
	    	wait1Msec(50);
	    	
	    	//while (joy1Btn(5) || joy1Btn(6) || joy1Btn(7) || joy1Btn(8)){}
	    	}
	    	}
	    	
	    	task main()
	    	{
	    	motor[RB] = motor[RF] = 0;
	    	motor[LB] = motor[LF] = 0;
	    	
	    	while(true)
	    	{
	    	getJoystickSettings(joystick);
	    	accurate_turn();
	    	joy_accomplish();
	    	}
	    	}
	    	
	    	\end{verbatim} 
	    }
	    % % % % % % % % % % %
	    \item Due to testing it was discovered, that optimal course speed to turn speed proportion varies non-linearly from one speed mode to another. So, it's more preferable to set speed mode by exact values of both speed parameters instead of common coefficient. In addition, it was decided to reduce the number of sectors on main stick's from 8 to 6 because 2 sectors were not in use %(fig. 2) 
	    .\newline
	    Here is the changed fragment of the code: \newline
	    {\small
	    \begin{verbatim}
  	    int joy_result()
  	    {
  		int horison_in = joystick.joy1_x1;
  		int vertical_in = joystick.joy1_y1;
  		
  		int option = -1;
  		int sector = 0;
  		
  		//determining sector
  		if (vertical_in > 10) sector += 1;
  		if (vertical_in < -10) sector += 4;
  		if (horison_in > 80) sector += 2;
  		if (horison_in < -80) sector += 8;
  		
  		//set option
  		if (sector == 1) option = 0;
  		if (sector == 3) option = 1;
  		if (sector == 2) option = 3; // sectors 3 & 5 won't be used
  		if (sector == 6) option = 2;
  		if (sector == 4) option = 4;
  		if (sector == 12) option = 6;
  		if (sector == 8) option = 5; ////
  		if (sector == 9) option = 7;
  		
  		return option;
  	    }
        \end{verbatim} 
        }
        
        \item Then the program was rewritten in Android Studio.
  \end{enumerate*}
  
  
\fillpage
