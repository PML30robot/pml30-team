\section{Introduction}

\subsection{How to read this book}

The book consists of 4 chapters. \newline
\begin{enumerate*}
    \item In the Introduction it is represented information about our team, our instructors and sponsors. 
    
    %At the end of introduction you can find short dictionary which contains specific terminology used in our book. It is needed to clarify the meanings of some terms that can be interpreted by the reader ambiguous.
    
    \item In the Business plan it is represented information about our sponsors' support and our budget.\newline
    
    \item The Engineering chapter consists of two parts. 
    \begin{itemize*}
    	\item The consequence of meetings, which show our progress in elaborating. This part includes sections 1 to 4.
    	
    	\item Documentations for each module and tele-op and autonomous programs. These materials are available in sections 5 and 6.
    	
    \end{itemize*}
    
    This approach allows to show the engineering process from two sides: development of the robot in general and development of each module in particular.
    
    The last section of Engineering chapter is "Key summary". It contains conclusive abilities of our robot in the game.
    
    \item Appendix includes a number of additional materials.
    \begin{itemize*}
    	\item The list of raw materials used in the robot.
    	
    	%\item The example of leaflet with our robot's characteristics that we intend to distribute among other teams to make them know about our abilities.
    	
    	\item The information list for judges.
    	
    \end{itemize*}
    
\end{enumerate*}

	
\fillpage	